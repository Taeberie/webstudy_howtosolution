\documentclass{article}
\author{Jeon Yongjin}
\usepackage{amsmath}
\usepackage{kotex}
\title{\LaTeX 해설 시연}
\date{}
\begin{document}
\maketitle

텍스트 중간에 수학 공식이 필요한 경우 \( y=x^2 \)의 형태를 사용합니다.


수학 공식이 별도의 공간을 필요로 하는 경우에는 다음과 같은 방식이 사용됩니다. \[ y=x^2 \]

계산 과정을 나타내고 싶을 때는
\[6x + 5 = 17\]
\[6x = 12\]
\[x = 2\]
이렇게 해도 되긴 합니다. 하지만 예쁘지가 않잖아요?

\begin{eqnarray*}
    6x + 5 &=& 12 \\
    6x &=& 7 \\
    x &=& \frac{7}{6}
\end{eqnarray*}
이게 일반적인 룰입니다. 훨씬 예쁘죠?

인라인 텍 구문은 줄맞춤 때문에 글씨크기가 작아집니다. \( \frac{3}{2} \) eqnarray 쓸 때랑 크기가 다르다는 점에 주목해 주세요.

잘 보인다고요? \(\frac{1}{1+\frac{1}{1+\frac{1}{2}}}\) 이건 어때요?

이럴 때 사용하는 명령이 displaystyle입니다. \( \displaystyle \frac{1}{2} \) 아까보다 훨씬 보기 좋죠?

특수문자같은 것들은, LaTeX Extension의 SNIPPET VIEW 기능을 적극 활용해 보세요. 정말 편합니다.
\(\alpha \psi E \lim_{x \to \infty} \prod \bigodot \gcd \subset \)

우리는 미적분학 과목의 해설을 쓰는 게 주 업무다 보니, 적분 코드 쓸 일이 상당히 많아요.


\end{document}