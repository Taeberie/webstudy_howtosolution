\documentclass{article}
\author{Jeon Yongjin}
\usepackage{amsmath}
\usepackage{kotex}
\title{해설 작성을 위한 \LaTeX 필수 문법 정리}
\date{}
\begin{document}
\maketitle

텍스트 중간에 수학 공식이 필요한 경우 \( y=x^2 \)의 형태를 사용합니다.

수학 공식이 별도의 공간을 필요로 하는 경우에는 다음과 같은 방식이 사용됩니다. \[ y=x^2 \]

계산 과정을 나타내고 싶을 때는
\[6x + 5 = 17\]
\[6x = 12\]
\[x = 2\]
이렇게 해도 되긴 합니다. 하지만 예쁘지가 않잖아요?

\begin{eqnarray*}
    6x + 5 &=& 12 \\
    6x &=& 7 \\
    x &=& \frac{7}{6}
\end{eqnarray*}
이게 일반적인 룰입니다. 훨씬 예쁘죠?

인라인 텍 구문은 줄맞춤 때문에 글씨크기가 작아집니다. \( \frac{3}{2} \) eqnarray 쓸 때랑 크기가 다르다는 점에 주목해 주세요.

잘 보인다고요? \(\frac{1}{1+\frac{1}{1+\frac{1}{2}}}\) 이건 어때요?

이럴 때 사용하는 명령이 displaystyle입니다. \( \displaystyle \frac{1}{2} \) 아까보다 훨씬 보기 좋죠?

특수문자같은 것들은, LaTeX Extension의 SNIPPET VIEW 기능을 적극 활용해 보세요. 정말 편합니다.
\(\alpha \psi E \lim_{x \to \infty} \prod \bigodot \gcd \subset \)

우리는 미적분학 과목의 해설을 쓰는 게 주 업무다 보니, 적분 코드 쓸 일이 상당히 많아요.

일전에 해설 시연 때 중간에 보여드렸던 코드를 다시 가져오겠습니다. 이거 참고하시면 에지간한 적분문제 해설 쓰는 데는 지장 없을 겁니다.

\begin{eqnarray*}
    \iiint 1 dV &=& \int_{0}^{2\pi} \int_{0}^{5} \int_{r}^{\sqrt{25-r^2}+5} r dz dr d\theta \\
    &=& 2\pi \int_{0}^{5} r\sqrt{25-r^2} + 5r -r^2 dr \\
    &=& 2\pi \left[ -\frac{1}{3}(25-r^2)^{\frac{3}{2}} + \frac{5}{2}r^2 - \frac{r^3}{3} \right]_{0}^{5} \\
    &=& 125\pi
\end{eqnarray*}

이제 자주 쓰게 되실 기능이 eqnarray랑 left right일 거에요. eqnarray는 뭐 방금 했으니까 됐고...
left right는 괄호의 크기를 괄호 안의 수식의 크기에 맞게 크기를 조정해 주는 기능입니다.
    
\[ 2\pi [-\frac{1}{3}(25-r^2)^{\frac{3}{2}} + \frac{5}{2}r^2 - \frac{r^3}{3}]_{0}^{5} \]
\[ 2\pi \left[ -\frac{1}{3}(25-r^2)^{\frac{3}{2}} + \frac{5}{2}r^2 - \frac{r^3}{3} \right]_{0}^{5} \]

비교해 보세요. 뭔지 아시겠죠?
left right는 꼭 짝이 맞아야 한다는 점을 참고하세요.

\[
f(x) =
\begin{cases} 
x^2 & \text{if } x \geq 0, \\
-x  & \text{if } x < 0.
\end{cases}
\]

이거도 종종 씁니다. 참고하세요.

아 그리고! 함수 이름 쓸 때 어지간하면 냅다 텍스트로 쓰지 말고 명령어를 쳐 주세요.

\(cos(x) \cos(x)\) 보시면 폰트가 다른 걸 확인하실 수 있습니다. 사소한 차이지만..은근 중요해요.

\end{document}